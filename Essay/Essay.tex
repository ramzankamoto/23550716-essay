\documentclass[preprint, 3p,
authoryear]{elsarticle} %review=doublespace preprint=single 5p=2 column
%%% Begin My package additions %%%%%%%%%%%%%%%%%%%

\usepackage[hyphens]{url}


\usepackage{graphicx}
%%%%%%%%%%%%%%%% end my additions to header

\usepackage[T1]{fontenc}
\usepackage{lmodern}
\usepackage{amssymb,amsmath}
% TODO: Currently lineno needs to be loaded after amsmath because of conflict
% https://github.com/latex-lineno/lineno/issues/5
\usepackage{lineno} % add
\usepackage{ifxetex,ifluatex}
\usepackage{fixltx2e} % provides \textsubscript
% use upquote if available, for straight quotes in verbatim environments
\IfFileExists{upquote.sty}{\usepackage{upquote}}{}
\ifnum 0\ifxetex 1\fi\ifluatex 1\fi=0 % if pdftex
  \usepackage[utf8]{inputenc}
\else % if luatex or xelatex
  \usepackage{fontspec}
  \ifxetex
    \usepackage{xltxtra,xunicode}
  \fi
  \defaultfontfeatures{Mapping=tex-text,Scale=MatchLowercase}
  \newcommand{\euro}{€}
\fi
% use microtype if available
\IfFileExists{microtype.sty}{\usepackage{microtype}}{}
\usepackage[]{natbib}
\bibliographystyle{elsarticle-harv}

\ifxetex
  \usepackage[setpagesize=false, % page size defined by xetex
              unicode=false, % unicode breaks when used with xetex
              xetex]{hyperref}
\else
  \usepackage[unicode=true]{hyperref}
\fi
\hypersetup{breaklinks=true,
            bookmarks=true,
            pdfauthor={},
            pdftitle={Risk Parity Portfolio Construction for South Africa},
            colorlinks=false,
            urlcolor=blue,
            linkcolor=magenta,
            pdfborder={0 0 0}}

\setcounter{secnumdepth}{5}
% Pandoc toggle for numbering sections (defaults to be off)


% tightlist command for lists without linebreak
\providecommand{\tightlist}{%
  \setlength{\itemsep}{0pt}\setlength{\parskip}{0pt}}




\usepackage{setspace}
\usepackage{ragged2e}
\usepackage{geometry}
\geometry{a4paper, textwidth=6in}
\renewcommand{\normalsize}{\fontsize{11pt}{13pt}\selectfont}
\justifying
\parskip=6pt
\makeatletter
\def\@journal{ }
\makeatother




\begin{document}


\begin{frontmatter}

  \title{Risk Parity Portfolio Construction for South Africa}
    \author[Stellenbosch University]{Ramzan Kamoto%
  \corref{cor1}%
  \fnref{1}}
   \ead{23550716@sun.ac.za} 
    \author[Stellenbosch University]{Nico Katzke%
  %
  }
   \ead{nfkatzke@gmail.com} 
      \affiliation[Stellenbosch University]{
    organization={Stellenbosch Economics
Department},addressline={Schumann Building, Bosman
Street},city={Stellenbosch},postcode={7600},state={Western
Cape},country={South Africa},}
    \cortext[cor1]{Corresponding author}
    \fntext[1]{Ramzan is a master's degree student at Stellenbosch
University. This project is in partial fulfillment of the Financial
Econometrics module.}
    \fntext[2]{Nico Katzke is the Head of Portfolios at Satrix and
part-time lecturer for Financial Econometrics and Financial Economics at
Stellenbosch University.}
  
  \begin{abstract}
  This study examines the effectiveness of risk parity portfolio
  construction in the South African equity market, using data from the
  FTSE/JSE Top 40 index from 2010 to 2024. A conceptual replication of
  the vanilla risk parity framework is conducted and applied to South
  Africa, comparing RP portfolios against the tangency portfolio in
  terms of allocation stability, risk distribution, and return
  performance. The results indicate that risk parity portfolios provide
  greater risk diversification and allocation stability, making them
  resilient in volatile markets. However, they may underperform in
  strong bull markets where high-risk assets generate outsized returns.
  \end{abstract}
    \begin{keyword}
    Risk Parity Portfolio \sep 
    Portfolio Optimization
  \end{keyword}
  
 \end{frontmatter}

\hypertarget{introduction}{%
\section{Introduction}\label{introduction}}

Portfolio optimization remains a fundamental challenge in financial
management, as investors seek to allocate capital efficiently across
different assets to balance risk and return. Traditional models, such as
Markowitz's (1952) mean-variance optimization (MVO) and the Capital
Asset Pricing Model (CAPM), have played a crucial role in shaping
investment strategies. However, these approaches heavily rely on
expected returns, which are notoriously difficult to estimate accurately
and are prone to instability.

Risk parity (RP), also known as equal risk contribution (ERC), has
emerged as an alternative approach to portfolio construction,
emphasizing the allocation of risk rather than capital. Unlike MVO,
which optimizes return for a given level of risk, RP portfolios ensure
that each asset contributes equally to overall portfolio risk. This
strategy gained popularity soon after the global financial crisis as
traditional methods of asset pricing faced scrutiny. The popularity grew
after \citet{maillard2010} showed that constructing portfolios with
equal risk contribution in mind led to better diversification outcomes
regardless of expected returns.

The objective of this paper is to examine the effectiveness of risk
parity portfolio construction in the South African equity market.
Specifically, we compare risk parity portfolios with the traditional
maximum Sharpe ratio (tangency) portfolio using a dataset of the
FTSE/JSE Top 40 companies from 2010 to 2024. By conducting a conceptual
replication of Tharis Souza's (2019) study on risk parity for FAANG
stocks, we assess whether risk parity offers superior risk-adjusted
returns and portfolio stability in an emerging market context. Through a
series of empirical tests, we evaluate the allocation dynamics, risk
stability, and return performance of RP portfolios against their
tangency counterparts.

\hypertarget{literature-review}{%
\section{Literature Review}\label{literature-review}}

\hypertarget{historical-evolution-of-portfolio-theory}{%
\subsection{Historical Evolution of Portfolio
Theory}\label{historical-evolution-of-portfolio-theory}}

The foundation of modern portfolio theory can be traced to
\citet{HM1952} work on mean variance optimization, which introduced the
concept of balancing expected returns against risk. The core concept of
MVO is to maximise expected return for a given level of risk, measured
by variance, by strategically allocating investments across different
assets to achieve the best possible risk-return trade-off for an
investor based on their risk tolerance \citep{panulo2014}. Although MVO
remains a powerful tool that is applied widely, it's practical
implementation has been criticised for its heavy reliance on expected
return estimates, which are prone to estimation errors and instability
\citep{best1991}.

Building on Markowitz's work, the Capital Asset Pricing Model,
introduced by \citet{sharpe1964} and \citet{mossin1966}, provided
further insight into asset pricing by linking expected returns to
systematic risk. While CAPM remains a cornerstone of financial theory,
its assumptions about market efficiency and normally distributed returns
have also been challenged.

The 2008 global financial crisis marked an important turning point that
led to many scholars and practitioners to strongly criticise traditional
risk-gain models, consequently, leading to wide scale adoption of a
different kind of portfolio selection strategy, one based on risk
allocation \citep{ararat2024}. One way to deal with the problems of MVO
is to use equal weighted portfolios (EW), where capital is distributed
evenly across assets. However, if the assets in the investment universe
have different instrinsic risk then the equal weighted portfolio does
not achieve total risk diversification \citep{ararat2024}.

\hypertarget{toward-risk-parity}{%
\subsection{Toward Risk Parity}\label{toward-risk-parity}}

A more prudent way of addressing risk is therefore risk parity, also
called equal risk contribution. A risk parity portfolio is constructed
to ensure that each asset contributes equally to the total risk of the
portfolio. Unlike EW portfolios, which allocates the same amount of
capital to each asset regardless of risk, RP adjusts asset weights based
on their individual volatility and correlation with other assets. This
ensures that no single asset dominates portfolio risk, leading to
improved diversification \citep{maillard2010}.

Mathematically, a risk parity portfolio is defined by solving the
following optimization problem:

\[
w_i\sigma_i = wj\sigma_j, \forall_{i,j}
\]

Where \(w_i\) represents the weight of asset I and \(\sigma_i\) is its
standard deviation. This framework ensures that assets with higher risk
receive lower weight, while lower risk assets are are allocated
proportionally more capital, creating a balanced risk structure across
the portfolio. Many methods exist when creating risk parity portfolios
\citep{feng2015}. Traditionally, the formulation of risk parity is a
least squares optimization problem, where the risk contribution of each
asset is compared to another assets risk contribution. The model thus
compares each asset with every other asset and reach an objective value
of 0 indicating that all risk contributions are the same
\citep{gambeta2020}.

The appeal of risk parity lies in its ability to distribute risk more
evenly across portfolio constituents, reducing reliance on fragile
return forecasts. However, this advantage is not without its theoretical
and empirical challenges. The assumption that equal risk contribution
leads to optimal diversification has been contested, particularly in
extreme market conditions. Traditional risk parity models rely on
volatility as a risk measure, which fails to capture asymmetries and
fat-tailed distributions present in financial markets \citep{braga2023}.
The increased research in extensions of ERC such as kurtosis-based risk
parity reflects a growing awareness that standard deviation alone may be
insufficient for portfolio optimization.

\citet{costa2019} argue that risk parity portfolios perform
inconsistently across market regimes, particularly during high
volatility periods. Their research introduces a Markov regime-switching
framework to address the instability in asset volatilities and
correlations. This approach suggests that the risk contributions of
assets should be dynamically adjusted based on the market state rather
than assuming static risk exposures. Such enhancements present a
compelling case for incorporating dynamic risk parity strategies that
react to macroeconomic conditions and shifts in risk premia. Another
concern is the over-concentatration of risk parity portfolios in low
volatility assets, particularly fixed income securities. Risk parity
portfolios seem to work well when interest rates are declining but
presents substantial risk when interest rates rise. \citet{panulo2014}
highlights the vulnerability of leveraged risk parity portfolios to
funding constraints, as liquidity dries up during financial crises. The
necessity for leverage in risk parity to maintain competitive returns
adds another layer of risk, raising questions about its suitability for
all investors.

Alternative portfolio construction methods provide contrasting views on
achieving diversification and risk control. MVO, despite its well
documented estimation errors, remains the theoretical benchmark for
efficient portfolio construction. With the advancements in machine
learning and Bayesian estimation techniques, some argue that return
forecasting has improved enough to mitigate drawbacks of MVO. Meanwhile,
hierarchical risk parity (HRP) has emerged as a more sophisticated
refinement of traditional risk parity \citep{nico2019}. It involves
clustering assets based on their correlation structures rather than
assuming homogeneity in risk contribution \citep{palit2024}. These
innovations demonstrate that risk parity, while valuable, is not
necessarily the optimal solution in all market conditions.

The ongoing debate between risk parity and alternative allocation
strategies underscores a broader dilemma in portfolio theory: should
investors prioritize robustness over theoretical efficiency? While risk
parity minimizes reliance on unreliable return estimates, it does not
eliminate estimation risk entirely. Volatility estimates and correlation
structures are equally prone to regime shifts and model instability. By
integrating higher order risk measures such as skewness and kurtosis,
researchers aim to refine risk parity's effectiveness in extreme market
conditions \citep{braga2023}. Future research should further explore
dynamic risk parity models that adjust allocation strategies in real
time based on changing market conditions. Factor-based risk parity,
which incorporates macroeconomic variables into the allocation process,
may also provide a more adaptive framework for portfolio construction.
Combining machine learning and risk parity has also proved to be useful,
as algorithmic techniques can enhance estimation stability and improve
responsiveness to financial shocks. Ultimately, risk parity remains an
essential strategy in portfolio construction. For the remainder of this
essay, I will advocate for risk parity portfolios using South African
equity data.

\hypertarget{data}{%
\section{Data}\label{data}}

In this paper, we explore whether a risk parity portfolio is more
efficient than maximum sharpe ratio or tangency portfolio in South
Africa. we conduct a conceptual replication of Tharis Souza's 2019
article,
``\textit{DIY Ray Dalio ETF: How to build your own Hedge Fund strategy with risk parity portfolios}'',
where he showed how to build a risk parity portfolio for FAANG
companies. In this paper, we follow the same principles, using the data
from the FTSE/JSE Top 40 Companies (J200) from January 2010 to June
2024. The data is available through Bloomberg. In order to stick close
to the article by Souza, which used 5 top companies in the USA, we
subset the number of equities by taking the top ten tickers on the last
date of the dataset. The portfolios are then constructed from the
universe of the following tickers: AGL, BTI, CFR, CPI, FSR, GFI, MTN,
NPN, PRX and SBK. Table one shows the descriptive statistics for these
tickers.

\hypertarget{methodology}{%
\section{Methodology}\label{methodology}}

For the purposes of this replication we use the J200 dataset to
construct a risk parity portfolio and a tangency portfolio. First, we
verify whether the risk parity portfolio is well specified, by plotting
the risk contribution and weight distribution of each asset. If the risk
contributions are all equal then the portfolio is well specified. Next
we compare the weight allocation of both portfolios to assess the weight
distribution among constituents. Thirdly, in order to determine the
stability of risk parity portfolios, we construct a rolling risk parity
portfolio and plot a stacked barchart of quarterly rebalanced weights.
Finally, the cumulative returns, daily returns and drawdowns are also
assessed and compared with performance from other portfolios in the
literature.

\hypertarget{rp-construction}{%
\subsection{RP Construction}\label{rp-construction}}

Let the marginal risk contribution of asset \(i\) be given by:

\[ 
\frac{\partial\sigma_{p}}{\partial{w}_{i}} = \frac{(\Sigma{w})_i}{\sigma_p}
\] Multiplying by \(w_i\) gives the total risk contribution (RC) :

\[
RC_i = w_i \frac{(\Sigma w)_i}{\sigma_p}
\] Since the total risk portfolio is:

\[
\sigma_p = \sqrt{w^\top \Sigma w},
\] the risk contribution simplifies to:

\[
RC_i = w_i (\Sigma w)_i.
\] In a risk parity portfolio, all assets contribute equally to risk,
this means :

\[
w_i (\Sigma w)_i = w_j (\Sigma w)_j, \quad \forall i, j.
\] Since this must hold for all pairs, we rewrite it as:

\[
w_i (\Sigma w)_i - w_j (\Sigma w)_j = 0, \quad \forall i, j,
\] This forms a homogenous system of equations and can be formulated as
a least squares objective problem as below

\[
\min_w \sum_{i=1}^n\sum_{j=1}^n \left( w_i (\Sigma w)_i - w_j (\Sigma w)_j \right)^2.\\
\] \[s.t \space \space \mathbf{1}^Tw= 1 \\\]

\[w \geq 0\]

The model will compare risk contribution of each asset with every other
asset and reach an objective value of 0 indicating all risk
contributions are the same.

\hypertarget{max-sharpe-ratio-construction}{%
\subsection{Max Sharpe Ratio
Construction}\label{max-sharpe-ratio-construction}}

The tangency portfolio maximizes the Sharpe ratio defined as:

\[
S(w) = \frac{w^\top\mu-r_f}{\sqrt{w^\top\Sigma{w}}}
\]

Where \(\mu\) is the vector of expected asset returns and \(r_f\) is the
risk free rate.

Since the Sharpe ratio is scale invariant, we solve the following
unconstrained optimization problem:

\[
\max_w \frac{w^\top (\mu - r_f \mathbf{1})}{\sqrt{w^\top \Sigma w}}.
\] Using Lagrangian optimization, the optimal tangency portfolio weights
before normalization are:

\[
w^* = \Sigma^{-1} (\mu - r_f \mathbf{1}).
\] To ensure that the portfolio is fully invested
\(w^\top \mathbf{1} = 1\), we normalize:

\[
w^* = \frac{\Sigma^{-1} (\mu - r_f \mathbf{1})}{\mathbf{1}^\top \Sigma^{-1} (\mu - r_f \mathbf{1})}.
\] This results in the tangency portfolio that leads to the highest
Sharpe ratio.

\hypertarget{discussion}{%
\section{Discussion}\label{discussion}}

The empirical results highlight several important findings regarding
risk parity portfolio performance in South Africa. First, the risk
contribution analysis confirms that the RP portfolio is well-specified,
with risk contributions distributed evenly across assets. This stands in
contrast to the maximum Sharpe ratio portfolio, where risk is
concentrated in a few high risk, high reward stocks.

Second, our weight distribution analysis reveals that risk parity
portfolios favor low-volatility assets more than the tangency portfolio,
which assigns disproportionate weights to assets with higher expected
returns. This finding aligns with existing literature, where RP
strategies tend to overweight assets with lower volatility. Moreover,
the rolling risk parity allocation over different time periods suggests
that RP portfolios maintain a relatively stable allocation structure,
whereas tangency portfolios would have exhibitted significant variation
in weight distributions due to changing return expectations. This
stability is particularly valuable in volatile market conditions, as it
minimizes exposure.

However, RP portfolios are not without their challenges. One notable
drawback is their tendency to underperform during strong bull markets,
where high-risk assets generate excess returns that RP portfolios do not
fully capture due to their focus on risk allocation rather than return
maximization. Additionally, the dependence of risk parity on volatility
estimates means that periods of sharp volatility shifts, such as during
financial crises, may lead to suboptimal allocations if not adjusted.

Finally, the literature shows that a risk parity portfolio would likely
be somewhere between an equal weight and mean variance portfolios
\citep{gambeta2020}. From this we can infer that the risk parity
portfolio would likely have higher cumulative returns and shallower
drawdowns than the tangency portfolio. This strengthens the argument for
their application in emerging market economies. To stress test this
portfolio structure further, it could have been useful to test other
packages on the same data as well as introduce a wider range of assets
to the investable universe. This would allow a thorough analysis of how
risk parity portfolios diversify across sectors, asset classes and
regulatory or macroeconomic contraints.

\hypertarget{conclusion}{%
\section{Conclusion}\label{conclusion}}

The empirical analysis shows that RP portfolios provide superior
diversification by ensuring an equal distribution of risk across assets,
reducing concentration in high-risk stocks, and maintaining allocation
stability over time. Given these findings, a dynamic risk parity
approach, which adjusts allocations based on changing market conditions,
such as a markov regime switching risk parity portfolio, may further
enhance performance. Incorporating factors such as macro-driven risk
assessments or machine learning techniques could improve risk parity's
adaptability to different market environments.

Future research should explore how hierarchical risk parity (HRP), which
clusters assets based on correlation structures, compares with
traditional RP in South African financial markets. Additionally,
integrating alternative risk measures, such as skewness and kurtosis,
may refine portfolio construction to better address tail-risk events.
Furthermore, incorporating different asset classes would illuminate how
risk parity construction diversifies across sectors and asset classes.

Ultimately, while risk parity does not necessarily outperform in all
market conditions, it remains a valuable tool for investors seeking
stability, diversification, and reduced dependence on unreliable return
forecasts. Its effectiveness in the South African context underscores
the need for risk-based allocation strategies in emerging markets, where
structural volatility and market inefficiencies pose unique challenges
to traditional investment models.

\hypertarget{appendix}{%
\section{Appendix}\label{appendix}}

\newpage

\renewcommand\refname{References}
\bibliography{mybibfile.bib}


\end{document}
